% ******************************************************************************%
%                                                                               % 
%         MCD_FancyCron.tex for FancyCron Project specification                 %
%            Made by: Eugène NGONTANG <ngonta_e@epitech.net>                    %
%                    engontan@bouyguestelecom.fr                                %
%                                                                               % 
% ******************************************************************************%

\documentclass{bouygues-fr}
%%\usepackage{sidecap}
\usepackage{supertabular}
\usepackage{longtable}

% ******************************************************************************%
%                                                                               % 
%                                 Prologue                                      %
%                                                                               % 
% ******************************************************************************%

\begin{document}



\title{\texttt{FancyCron}}
\subtitle{Analyse fonctionnelle}

\member{Eugène NGONTANG}{ngonta_e@epitech.net}
\member{DSI/DOQS/PFI Bougues Telecom}{engontan@bouyguestelecom.fr}

%%\newpage

\summary
{
  FancyCron est un logiciel de planification et d'exécution de tâches, pour système Linux et Windows. Le projet est né d'un besoin des administrateurs système à la DSI FAI de Bouygues Telecom.

  Ces derniers se servent du gestionnaire de tâches Linux Cron, pour l'automatisation des services réguliers sur l'ensemble des serveurs de leurs infrastructures.\\
Ils se trouvent limités dans le suivi de ces tâches, notamment par le manque d'informations pertinentes sur le déroulement de l'exécution des tâches de bout en bout.

  FancyCron viendra donc répondre à une problématique de planning production, de reporting et de gestion centralisée des tâches sur leurs environnements techniques. 

  Ce document constitue le document d'analyse fonctionnelle du projet. Il décrit les données du système à développer et le modèle conceptuel de données associé.
}

\maketitle

\tableofcontents

\renewcommand{\labelitemi}{$\bullet$}
\renewcommand{\labelitemii}{$\circ$}

\newpage
% ******************************************************************************%
%                                                                               % 
%                                 Les données                                   %
%                                                                               % 
% ******************************************************************************%

\chapter{Données}

Comme types de données/entités nous aurons:
\begin{itemize}
  \item  \textcolor{violet}{\texttt{Les groupes métier}}, qui représentent des serveurs(hôtes) regroupé d'une même catégorie
  \item  \textcolor{violet}{\texttt{Les hotes}}, qui sont les serveurs du parc sur lesquels tournent des \textcolor{violet}{texttt{crontab}}
  \item  \textcolor{violet}{\texttt{Les tâches}}. Une tâche est un programme s'exécutant sur un serveur, suivant un calendrier défini
  \item  \textcolor{violet}{\texttt{Les calendriers (schedule)}}. Ils définissent le cycle selon lequel une tâche doit s'exécuter
  \item  \textcolor{violet}{\texttt{Les enchaînements (worflows)}}. Ils gèrent enchaînements et dépendances entre les tâches. La fonctionnalité n'est pas implémentée pour la première version du logiciel
  \item  \textcolor{violet}{\texttt{Les utilisateurs}}. Ce sont les utilisateurs au sens système(bash sous linux par exemple)
  \item  \textcolor{violet}{\texttt{Les dépendances}}. Ce sont des tâches dont dépendent d'autres tâches
  \item  \textcolor{violet}{\texttt{Les historiques de tâches}}. Ils stockent journalièrement les informations sur les tâches exécutées.
  \item  \textcolor{violet}{\texttt{Les comptes}}. Ils s'agit ici des comptes utilisateurs du logiciel même\\
    Ils n'entrent pas réellement dans le système à mettre en place, et servent simplement à identifier qui peut ou non utiliser l'application.\\
    A chaque utilisateur est associé un niveau/profile, lui donnant certains droits sur l'édition des paramètres du système.
\end{itemize}

\vspace{20pt}
\chapter{Propriétés des entités et dépendances fonctionnelles}

\section{Les groupes métier}
Un groupe métier est caractérisé par :
\begin{itemize}
\item son identifiant, qui le représente de façon unique sur le système
\item son nom
\end{itemize} 

\section{Les hôtes}
Un hôte est caractérisé par :
\begin{itemize}
\item son identifiant, qui le représente de façon unique sur le système
\item son nom
\item son adresse ip
\end{itemize} 
 
Il est associé à un groupe métier par un \textcolor{violet}{\texttt{identifiant de groupe métier}}

\section{Les utilisateurs}
Un utilisateur est caractérisé par :
\begin{itemize}
\item son identifiant, qui le représente de façon unique sur le système
\item son nom
\item son groupe(groupe au sens système)
\end{itemize} 

Il est associé à un hôte par un \textcolor{violet}{\texttt{identifiant d'hôte}}

\section{Les schedules}
Un schedule est caractérisée par :
\begin{itemize}
\item son identifiant, qui le représente de façon unique sur le système
\item son nom
\item sa description
\item une valeur de minute
\item une valeur d'heure
\item une valeur de jour
\item une valeur de mois
\item une valeur d'année
\end{itemize} 

Ces cinq dernières valeur seront représentées suivant un format et une convention à adopter dans \textcolor{violet}{\texttt{FancyCron}}

\section{Les tâches}
Une tâche est caractérisée par :
\begin{itemize}
\item son identifiant, qui la représente de façon unique sur le système
\item son nom
\item son statut(activé ou désactivé)
\item sa description
\item la commande à exécuter
\item son pramétrage(arguments + options de la commande)
\item le résultat de la dernière exécution
\item la date de la dernière exécution
\item la date de la prochaine exécution. Ce paramètre est mis à jour chaque fois que la tâche subit un décalage
\item l'adresse du fichier de log
\end{itemize} 
 
Elle est associée à :
\begin{itemize}
\item un utilisateur par un \textcolor{violet}{\texttt{identifiant d'utilisateur}},
\item un hôte par un \textcolor{violet}{\texttt{identifiant d'hôte}},
\item un workflow par un \textcolor{violet}{\texttt{identifiant de workflow}}.
\end{itemize}

\section{Les historiques}
Un historique est caractérisé par :
\begin{itemize}
\item son identifiant, qui le représente de façon unique sur le système
\item la dâte de début de tâche sur le système local où elle a été planifiée
\item la dâte de fin de tâche sur le système local où elle a été planifiée
\item le statut d'exécution de la tâche
\item le nombre d'occurence d'exécution de la tâche pour la date correspondant à l'entrée de la table
\end{itemize}

Il est associé à une tâche par un \textcolor{violet}{\texttt{identifiant de tâche}}.

\section{Les dépendances}
Un dépendance est caractérisée par :
\begin{itemize}
\item son identifiant, qui le représente de façon unique sur le système
\item son nom de tâche(tâche dépendante)
\item le critère de dépendance : terminé(exécuté et terminé), terminé avec erreur, terminer avec succès
\end{itemize} 

Elle est associée à une tâche par un \textcolor{violet}{\texttt{identifiant de tâche}}, qui représente en fait la tâche constituant la dépendance.

\section{Les workflows}
Un workflow est caractérisé par une liste de dépendances

\subsection{Les comptes}
Un compte est caractérisé par :
\begin{itemize}
\item son identifiant, qui le représente de façon unique sur le système
\item son nom
\item son mot de passe
\item son profile
\end{itemize} 

Vous trouverez le modèle conceptuel de données à \href{https://mcp-d.admin.dolmen.bouyguestelecom.fr/fancycron/DOCS/MCD_FC.pdf}{cette adresse} et le modèle physique de données \href{https://mcp-d.admin.dolmen.bouyguestelecom.fr/fancycron/DOCS/MPD_FC.pdf}{ici}

  % ******************************************************************************%
\end{document}